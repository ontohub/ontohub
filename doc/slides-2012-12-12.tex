\documentclass[german]{beamer}

\mode<presentation>
{
  \usetheme{Warsaw}
  \setbeamercovered{transparent}
}

\usepackage{latexsym}
\usepackage{url}
\usepackage{appendixnumberbeamer}

%\usepackage{german}
\usepackage[latin1]{inputenc} %% Umlaute

\usepackage{stmaryrd,mathbbol,mathrsfs}
\usepackage{amsmath,amstext,amssymb,amsopn}
\usepackage{exscale,calc,ifthen,array}
\usepackage{float}
\usepackage{overpic}


\usepackage{xspace,alltt}
\usepackage{amssymb}
\usepackage[ps,dvips]{xy}
\xyoption{v2}
\xyoption{line}
\xyoption{curve}

%\definecolor{red}{rgb}{1,0,0} % red
%\definecolor{green}{rgb}{0,1,0} % green
%\definecolor{blue}{rgb}{0,0,1} % blue

\newcommand{\emphc}[1]{{\color{red}{#1}}}

\newcommand{\red}[1]{{\color{red}{#1}}}
\newcommand{\green}[1]{{\color{green}{#1}}}
\newcommand{\blue}[1]{{\color{blue}{#1}}}
\newcommand{\white}[1]{{\color{white}{#1}}}


\newlength{\PicSize}
\newcommand{\CASL}{\textsc{CASL}\xspace}
\newcommand{\HetCASL}{\textmd{\textsc{HetCasl}}\xspace }
\newcommand{\Hets}{\textmd{\textsc{Hets}}\xspace }

\newcommand{\mystrut}[1]{\rule[#1]{0cm}{0.1cm}}


%% Note: If title has linebreaks, you must use a short title without 
%%       linebreaks, otherwise you'll get funny LaTeX errors.
\title{OntoHub-Perspektivtag}

%% Note about linebreaks applies here as well.
\author{The Ontohub team}

\date{12.12.12}

\logo{\includegraphics[height=0.8cm]{DFKI-Logo.jpg}
\hspace{5.5cm}
\includegraphics[height=0.8cm]{sfb_tr8_logo.pdf}}

\begin{document}

\maketitle

\begin{frame}
\frametitle{Zeitplan}
\begin{tabular}{|l|p{9cm}|}\hline
10h & Vorstellung / Begr��ung / Motivation / Know-How\\\hline
10.30h & Vorstellung von Ontohub und Hintergrund (OntoIOp ISO-Standard, OOR community, Hets community)\\\hline
11h & Ontohub.org Webportal-Demo\\\hline
11.30h & Ontohub-Architektur und -Datenmodell\\\hline
13h & Essen\\\hline
14h & Ziele\\\hline
14.30h & Motivation / Know-How / Arbeitsbereiche aufteilen\\\hline
15h & Arbeitsweise (regelm��ige Treffen, Programmiersessions, github, issue tracker)\\\hline
15.30h & Fragen zu Entwicklungsumgebung aufsetzen\\\hline
ca. 16h & Ende\\\hline
\end{tabular}
\end{frame}

\begin{frame}
\frametitle{Ontologien}
\begin{itemize}
\item 
\item 
\item 
\end{itemize}
\end{frame}

\begin{frame}
\frametitle{Ontology Integration and Interoperability (OntoIOp)}
\begin{itemize}
\item ISO Standard 17347 (derzeit: working draft)
\item Distributed Ontology Language (DOL)
\begin{itemize}
\item OWL (Web Ontology Language)
\item RDF (Resource Description Framework)
\item EER (Enhanced Entity-Relationship Diagrams)
\item Common Logic
\item UML (Unified Modeling Language)
\item \ldots
\end{itemize}
\item Community von 10-15 Experten
\item 14-t�gig Telefonkonferenzen
\end{itemize}
\end{frame}

\begin{frame}
\frametitle{Heterogeneous Tool Set (Hets)}
\begin{itemize}
\item Logik-Backend f�r OntoHub
\item Kann alle Sprachen parsen und analysieren
\item Schnittstelle zu ca.\ 15 Beweiswerkzeugen
\item \url{http://www.dfki.de/cps/hets}
\end{itemize}
\end{frame}

\begin{frame}
\frametitle{Ontohub-Architektur und -Datenmodell}
\begin{itemize}
\item 
\item 
\item 
\end{itemize}
\end{frame}

\begin{frame}
\frametitle{OntoHub-Ziele}
\begin{itemize}
\item Funktionalit�t sicherstellen (Tests schreiben)
\item CRUD-Funktionalit�t f�r Logiken, Translations, \ldots
\begin{itemize}
\item Logik-Graphen aus RDF-Registry einlesen
\end{itemize}
\item Graphen-Darstellung von Ontologien+Links und Logikgraphen
\item Reasoning
\item Git-Backend (Richtung verteilte Entwicklung)
\item langfirstiges Ziel: OOR-Architektur 
\end{itemize}
\end{frame}


\begin{frame}
\frametitle{Git-Backend}
\begin{itemize}
\item Neues Repository-Model, das gleichzeitig Namespace f�r Ontologien darstellt  
\item unterliegend Git- oder svn-Repositories
\item Berechtigungen werden nicht mehr pro Ontologie, sondern pro Repository vergeben
\item Nur Inhalte im "master"-Branch ber�cksichtigen 
\end{itemize}
\end{frame}

\begin{frame}
\frametitle{Arbeitsweise}
\begin{itemize}
\item 
\item 
\item 
\end{itemize}
\end{frame}

\end{document}

\begin{frame}
\frametitle{}
\begin{itemize}
\item 
\item 
\item 
\end{itemize}
\end{frame}


Zielskizzen



Speichern: Wer hat was wann wohin gepusht?
server side hooks
http://git-scm.com/book/en/Customizing-Git-Git-Hooks#Server-Side-Hooks
Suche?
Evtl. NoSQL (MongoDB, CouchDB)



\begin{frame}
\frametitle{}
\begin{itemize}
\item 
\item 
\item 
\end{itemize}
\end{frame}
\begin{frame}
\frametitle{}
\begin{itemize}
\item 
\item 
\item 
\end{itemize}
\end{frame}
\begin{frame}
\frametitle{}
\begin{itemize}
\item 
\item 
\item 
\end{itemize}
\end{frame}
\begin{frame}
\frametitle{}
\begin{itemize}
\item 
\item 
\item 
\end{itemize}
\end{frame}
 	
